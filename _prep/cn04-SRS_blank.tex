% Options for packages loaded elsewhere
\PassOptionsToPackage{unicode}{hyperref}
\PassOptionsToPackage{hyphens}{url}
\PassOptionsToPackage{dvipsnames,svgnames,x11names}{xcolor}
%
\documentclass[
  letterpaper,
  DIV=11,
  numbers=noendperiod]{scrartcl}

\usepackage{amsmath,amssymb}
\usepackage{lmodern}
\usepackage{iftex}
\ifPDFTeX
  \usepackage[T1]{fontenc}
  \usepackage[utf8]{inputenc}
  \usepackage{textcomp} % provide euro and other symbols
\else % if luatex or xetex
  \usepackage{unicode-math}
  \defaultfontfeatures{Scale=MatchLowercase}
  \defaultfontfeatures[\rmfamily]{Ligatures=TeX,Scale=1}
\fi
% Use upquote if available, for straight quotes in verbatim environments
\IfFileExists{upquote.sty}{\usepackage{upquote}}{}
\IfFileExists{microtype.sty}{% use microtype if available
  \usepackage[]{microtype}
  \UseMicrotypeSet[protrusion]{basicmath} % disable protrusion for tt fonts
}{}
\makeatletter
\@ifundefined{KOMAClassName}{% if non-KOMA class
  \IfFileExists{parskip.sty}{%
    \usepackage{parskip}
  }{% else
    \setlength{\parindent}{0pt}
    \setlength{\parskip}{6pt plus 2pt minus 1pt}}
}{% if KOMA class
  \KOMAoptions{parskip=half}}
\makeatother
\usepackage{xcolor}
\setlength{\emergencystretch}{3em} % prevent overfull lines
\setcounter{secnumdepth}{-\maxdimen} % remove section numbering
% Make \paragraph and \subparagraph free-standing
\ifx\paragraph\undefined\else
  \let\oldparagraph\paragraph
  \renewcommand{\paragraph}[1]{\oldparagraph{#1}\mbox{}}
\fi
\ifx\subparagraph\undefined\else
  \let\oldsubparagraph\subparagraph
  \renewcommand{\subparagraph}[1]{\oldsubparagraph{#1}\mbox{}}
\fi

\usepackage{color}
\usepackage{fancyvrb}
\newcommand{\VerbBar}{|}
\newcommand{\VERB}{\Verb[commandchars=\\\{\}]}
\DefineVerbatimEnvironment{Highlighting}{Verbatim}{commandchars=\\\{\}}
% Add ',fontsize=\small' for more characters per line
\newenvironment{Shaded}{}{}
\newcommand{\AlertTok}[1]{\textcolor[rgb]{1.00,0.33,0.33}{\textbf{#1}}}
\newcommand{\AnnotationTok}[1]{\textcolor[rgb]{0.42,0.45,0.49}{#1}}
\newcommand{\AttributeTok}[1]{\textcolor[rgb]{0.84,0.23,0.29}{#1}}
\newcommand{\BaseNTok}[1]{\textcolor[rgb]{0.00,0.36,0.77}{#1}}
\newcommand{\BuiltInTok}[1]{\textcolor[rgb]{0.84,0.23,0.29}{#1}}
\newcommand{\CharTok}[1]{\textcolor[rgb]{0.01,0.18,0.38}{#1}}
\newcommand{\CommentTok}[1]{\textcolor[rgb]{0.42,0.45,0.49}{#1}}
\newcommand{\CommentVarTok}[1]{\textcolor[rgb]{0.42,0.45,0.49}{#1}}
\newcommand{\ConstantTok}[1]{\textcolor[rgb]{0.00,0.36,0.77}{#1}}
\newcommand{\ControlFlowTok}[1]{\textcolor[rgb]{0.84,0.23,0.29}{#1}}
\newcommand{\DataTypeTok}[1]{\textcolor[rgb]{0.84,0.23,0.29}{#1}}
\newcommand{\DecValTok}[1]{\textcolor[rgb]{0.00,0.36,0.77}{#1}}
\newcommand{\DocumentationTok}[1]{\textcolor[rgb]{0.42,0.45,0.49}{#1}}
\newcommand{\ErrorTok}[1]{\textcolor[rgb]{1.00,0.33,0.33}{\underline{#1}}}
\newcommand{\ExtensionTok}[1]{\textcolor[rgb]{0.84,0.23,0.29}{\textbf{#1}}}
\newcommand{\FloatTok}[1]{\textcolor[rgb]{0.00,0.36,0.77}{#1}}
\newcommand{\FunctionTok}[1]{\textcolor[rgb]{0.44,0.26,0.76}{#1}}
\newcommand{\ImportTok}[1]{\textcolor[rgb]{0.01,0.18,0.38}{#1}}
\newcommand{\InformationTok}[1]{\textcolor[rgb]{0.42,0.45,0.49}{#1}}
\newcommand{\KeywordTok}[1]{\textcolor[rgb]{0.84,0.23,0.29}{#1}}
\newcommand{\NormalTok}[1]{\textcolor[rgb]{0.14,0.16,0.18}{#1}}
\newcommand{\OperatorTok}[1]{\textcolor[rgb]{0.14,0.16,0.18}{#1}}
\newcommand{\OtherTok}[1]{\textcolor[rgb]{0.44,0.26,0.76}{#1}}
\newcommand{\PreprocessorTok}[1]{\textcolor[rgb]{0.84,0.23,0.29}{#1}}
\newcommand{\RegionMarkerTok}[1]{\textcolor[rgb]{0.42,0.45,0.49}{#1}}
\newcommand{\SpecialCharTok}[1]{\textcolor[rgb]{0.00,0.36,0.77}{#1}}
\newcommand{\SpecialStringTok}[1]{\textcolor[rgb]{0.01,0.18,0.38}{#1}}
\newcommand{\StringTok}[1]{\textcolor[rgb]{0.01,0.18,0.38}{#1}}
\newcommand{\VariableTok}[1]{\textcolor[rgb]{0.89,0.38,0.04}{#1}}
\newcommand{\VerbatimStringTok}[1]{\textcolor[rgb]{0.01,0.18,0.38}{#1}}
\newcommand{\WarningTok}[1]{\textcolor[rgb]{1.00,0.33,0.33}{#1}}

\providecommand{\tightlist}{%
  \setlength{\itemsep}{0pt}\setlength{\parskip}{0pt}}\usepackage{longtable,booktabs,array}
\usepackage{calc} % for calculating minipage widths
% Correct order of tables after \paragraph or \subparagraph
\usepackage{etoolbox}
\makeatletter
\patchcmd\longtable{\par}{\if@noskipsec\mbox{}\fi\par}{}{}
\makeatother
% Allow footnotes in longtable head/foot
\IfFileExists{footnotehyper.sty}{\usepackage{footnotehyper}}{\usepackage{footnote}}
\makesavenoteenv{longtable}
\usepackage{graphicx}
\makeatletter
\def\maxwidth{\ifdim\Gin@nat@width>\linewidth\linewidth\else\Gin@nat@width\fi}
\def\maxheight{\ifdim\Gin@nat@height>\textheight\textheight\else\Gin@nat@height\fi}
\makeatother
% Scale images if necessary, so that they will not overflow the page
% margins by default, and it is still possible to overwrite the defaults
% using explicit options in \includegraphics[width, height, ...]{}
\setkeys{Gin}{width=\maxwidth,height=\maxheight,keepaspectratio}
% Set default figure placement to htbp
\makeatletter
\def\fps@figure{htbp}
\makeatother

\KOMAoption{captions}{tableheading}
\makeatletter
\@ifpackageloaded{tcolorbox}{}{\usepackage[many]{tcolorbox}}
\@ifpackageloaded{fontawesome5}{}{\usepackage{fontawesome5}}
\definecolor{quarto-callout-color}{HTML}{909090}
\definecolor{quarto-callout-note-color}{HTML}{0758E5}
\definecolor{quarto-callout-important-color}{HTML}{CC1914}
\definecolor{quarto-callout-warning-color}{HTML}{EB9113}
\definecolor{quarto-callout-tip-color}{HTML}{00A047}
\definecolor{quarto-callout-caution-color}{HTML}{FC5300}
\definecolor{quarto-callout-color-frame}{HTML}{acacac}
\definecolor{quarto-callout-note-color-frame}{HTML}{4582ec}
\definecolor{quarto-callout-important-color-frame}{HTML}{d9534f}
\definecolor{quarto-callout-warning-color-frame}{HTML}{f0ad4e}
\definecolor{quarto-callout-tip-color-frame}{HTML}{02b875}
\definecolor{quarto-callout-caution-color-frame}{HTML}{fd7e14}
\makeatother
\makeatletter
\makeatother
\makeatletter
\makeatother
\makeatletter
\@ifpackageloaded{caption}{}{\usepackage{caption}}
\AtBeginDocument{%
\ifdefined\contentsname
  \renewcommand*\contentsname{Table of contents}
\else
  \newcommand\contentsname{Table of contents}
\fi
\ifdefined\listfigurename
  \renewcommand*\listfigurename{List of Figures}
\else
  \newcommand\listfigurename{List of Figures}
\fi
\ifdefined\listtablename
  \renewcommand*\listtablename{List of Tables}
\else
  \newcommand\listtablename{List of Tables}
\fi
\ifdefined\figurename
  \renewcommand*\figurename{Figure}
\else
  \newcommand\figurename{Figure}
\fi
\ifdefined\tablename
  \renewcommand*\tablename{Table}
\else
  \newcommand\tablename{Table}
\fi
}
\@ifpackageloaded{float}{}{\usepackage{float}}
\floatstyle{ruled}
\@ifundefined{c@chapter}{\newfloat{codelisting}{h}{lop}}{\newfloat{codelisting}{h}{lop}[chapter]}
\floatname{codelisting}{Listing}
\newcommand*\listoflistings{\listof{codelisting}{List of Listings}}
\makeatother
\makeatletter
\@ifpackageloaded{caption}{}{\usepackage{caption}}
\@ifpackageloaded{subcaption}{}{\usepackage{subcaption}}
\makeatother
\makeatletter
\@ifpackageloaded{tcolorbox}{}{\usepackage[many]{tcolorbox}}
\makeatother
\makeatletter
\@ifundefined{shadecolor}{\definecolor{shadecolor}{rgb}{.97, .97, .97}}
\makeatother
\makeatletter
\makeatother
\ifLuaTeX
  \usepackage{selnolig}  % disable illegal ligatures
\fi
\IfFileExists{bookmark.sty}{\usepackage{bookmark}}{\usepackage{hyperref}}
\IfFileExists{xurl.sty}{\usepackage{xurl}}{} % add URL line breaks if available
\urlstyle{same} % disable monospaced font for URLs
\hypersetup{
  pdftitle={Simple Random Sampling},
  pdfauthor={YOUR NAME HERE},
  colorlinks=true,
  linkcolor={blue},
  filecolor={Maroon},
  citecolor={Blue},
  urlcolor={Blue},
  pdfcreator={LaTeX via pandoc}}

\title{Simple Random Sampling}
\author{YOUR NAME HERE}
\date{2/20/23}

\begin{document}
\maketitle
\ifdefined\Shaded\renewenvironment{Shaded}{\begin{tcolorbox}[frame hidden, sharp corners, breakable, boxrule=0pt, borderline west={3pt}{0pt}{shadecolor}, interior hidden, enhanced]}{\end{tcolorbox}}\fi

\begin{verbatim}
Error in library(survey): there is no package called 'survey'
\end{verbatim}

\hypertarget{introduction}{%
\section{Introduction}\label{introduction}}

This section will aim to answer the following questions:

\begin{itemize}
\tightlist
\item
  How can we draw a simple random sample of observations from a data set
  both with, and without replacement:
\item
  What is the finite population correction (fpc) and why do we need it?
\item
  What are sampling weights, why do we need them, and how are they
  created?
\item
  How do we calculate parameter estimates from an SRS that account for
  both the fpc and sampling weights?
\end{itemize}

\hypertarget{drawing-a-simple-random-sample}{%
\section{Drawing a Simple Random
Sample}\label{drawing-a-simple-random-sample}}

Recall there are two ways to draw simple random samples, with and
without replacement.

\begin{tcolorbox}[enhanced jigsaw, toptitle=1mm, breakable, colframe=quarto-callout-important-color-frame, colbacktitle=quarto-callout-important-color!10!white, left=2mm, titlerule=0mm, bottomtitle=1mm, title=\textcolor{quarto-callout-important-color}{\faExclamation}\hspace{0.5em}{Definition: Simple random sample (SRSWR) with replacement:}, bottomrule=.15mm, rightrule=.15mm, arc=.35mm, toprule=.15mm, colback=white, opacityback=0, leftrule=.75mm, coltitle=black, opacitybacktitle=0.6]

A SRSWR of size \(n\) from a population of size \(N\) can be thought of
as drawing \(n\) independent samples of size 1. Each unit has the same
probability of selection: \(\delta_{i} = \frac{1}{N}\)

The procedure is repeated until the sample has \(n\) units, which may
include duplicates.

\end{tcolorbox}

\begin{tcolorbox}[enhanced jigsaw, toptitle=1mm, breakable, colframe=quarto-callout-important-color-frame, colbacktitle=quarto-callout-important-color!10!white, left=2mm, titlerule=0mm, bottomtitle=1mm, title=\textcolor{quarto-callout-important-color}{\faExclamation}\hspace{0.5em}{Definition: Simple random sample (SRS) without replacement:}, bottomrule=.15mm, rightrule=.15mm, arc=.35mm, toprule=.15mm, colback=white, opacityback=0, leftrule=.75mm, coltitle=black, opacitybacktitle=0.6]

A SRS of size \(n\) is selected so that every possible subset of \(n\)
distinct units in the population has the same probability of being
selected as the sample. There are \(\binom{N}{n}\) possible samples,
resulting in a selection probability for an individual unit
\(\delta_{i} = \frac{n}{N}\). (See Lohr 2.10 and Appendix A for
derivation)

\end{tcolorbox}

\hypertarget{intentionality-in-sampling}{%
\subsection{Intentionality in
sampling}\label{intentionality-in-sampling}}

Random does not mean haphazard, contrary it's actually quite
intentional. Avoid selecting a sample that you ``feel'' is random or
representative of the population. These practices can lead to bias and
lack of generalizability.

To avoid the haphazard nature of ``blindly choosing'', or worse looking
at what was sampled and changing it because ``it doesn't look random
enough'', we use techniques that leverage pseudo-random number
generating algorithms.

\begin{tcolorbox}[enhanced jigsaw, toptitle=1mm, breakable, colframe=quarto-callout-important-color-frame, colbacktitle=quarto-callout-important-color!10!white, left=2mm, titlerule=0mm, bottomtitle=1mm, title=\textcolor{quarto-callout-important-color}{\faExclamation}\hspace{0.5em}{Process}, bottomrule=.15mm, rightrule=.15mm, arc=.35mm, toprule=.15mm, colback=white, opacityback=0, leftrule=.75mm, coltitle=black, opacitybacktitle=0.6]

\begin{enumerate}
\def\labelenumi{\arabic{enumi}.}
\tightlist
\item
  Generate a list of all observational units in the population (sampling
  frame).
\item
  Assign each observational unit a unique number, from 1 to the size of
  the sampling frame \(N\).
\item
  Use a computer to draw \(n\) numbers from 1 to \(N\) without
  replacement.
\item
  Subset the data to keep only the selected rows.
\end{enumerate}

\end{tcolorbox}

\hypertarget{drawing-a-srs-using-a-computer}{%
\subsection{Drawing a SRS using a
computer}\label{drawing-a-srs-using-a-computer}}

Previously we saw how to use the \texttt{sample} function to accomplish
this. An alternative is to use the functions \texttt{srswor} or
\texttt{srswr} from the \texttt{sampling} package. Each have their pros
and cons, so we explore both.

\begin{tcolorbox}[enhanced jigsaw, toptitle=1mm, breakable, colframe=quarto-callout-tip-color-frame, colbacktitle=quarto-callout-tip-color!10!white, left=2mm, titlerule=0mm, bottomtitle=1mm, title=\textcolor{quarto-callout-tip-color}{\faLightbulb}\hspace{0.5em}{Example: Sampling from Dr.~D's animal names.}, bottomrule=.15mm, rightrule=.15mm, arc=.35mm, toprule=.15mm, colback=white, opacityback=0, leftrule=.75mm, coltitle=black, opacitybacktitle=0.6]

\begin{Shaded}
\begin{Highlighting}[]
\FunctionTok{set.seed}\NormalTok{(}\DecValTok{4134}\NormalTok{)}
\NormalTok{my.animals }\OtherTok{\textless{}{-}} \FunctionTok{c}\NormalTok{(}\StringTok{"Toki"}\NormalTok{, }\StringTok{"Meka"}\NormalTok{, }\StringTok{"Riley"}\NormalTok{, }\StringTok{"TJ"}\NormalTok{, }\StringTok{"Dodger"}\NormalTok{, }\StringTok{"DC"}\NormalTok{, }\StringTok{"Sid"}\NormalTok{, }\StringTok{"Spike"}\NormalTok{)}
\end{Highlighting}
\end{Shaded}

\end{tcolorbox}

When using \texttt{sample()} the vector of data that you want to sample
from is provided as the first argument, and what is returned is the
values in that vector.

\begin{Shaded}
\begin{Highlighting}[]
\FunctionTok{sample}\NormalTok{()}
\end{Highlighting}
\end{Shaded}

\begin{verbatim}
Error in sample(): argument "x" is missing, with no default
\end{verbatim}

\begin{Shaded}
\begin{Highlighting}[]
\FunctionTok{sample}\NormalTok{()}
\end{Highlighting}
\end{Shaded}

\begin{verbatim}
Error in sample(): argument "x" is missing, with no default
\end{verbatim}

The functions \texttt{srswr} and \texttt{srswor} draw SRS with, and
without replacement respectively. Each take two arguments: \(n\) the
sample size, and \(N\) the population size, and what is returned is a
vector of length \(N\) that indicates how many times that position in
the vector is selected.

\begin{Shaded}
\begin{Highlighting}[]
\FunctionTok{set.seed}\NormalTok{(}\DecValTok{4134}\NormalTok{)}
\NormalTok{(choose.these.wor }\OtherTok{\textless{}{-}}\NormalTok{ ) }
\NormalTok{(choose.these.wr }\OtherTok{\textless{}{-}}\NormalTok{ ) }
\end{Highlighting}
\end{Shaded}

\begin{verbatim}
Error: <text>:2:22: unexpected ')'
1: set.seed(4134)
2: (choose.these.wor <- )
                        ^
\end{verbatim}

Then the \texttt{getdata()} function is used to extract the values from
the population the indicated number of times.

\begin{Shaded}
\begin{Highlighting}[]
\FunctionTok{getdata}\NormalTok{()}
\end{Highlighting}
\end{Shaded}

\begin{verbatim}
Error in is.data.frame(data): argument "data" is missing, with no default
\end{verbatim}

\begin{Shaded}
\begin{Highlighting}[]
\FunctionTok{getdata}\NormalTok{()}
\end{Highlighting}
\end{Shaded}

\begin{verbatim}
Error in is.data.frame(data): argument "data" is missing, with no default
\end{verbatim}

This method can be advantageous when drawing with replacement from a
large dataset.

\begin{tcolorbox}[enhanced jigsaw, toptitle=1mm, breakable, colframe=quarto-callout-warning-color-frame, colbacktitle=quarto-callout-warning-color!10!white, left=2mm, titlerule=0mm, bottomtitle=1mm, title={You try it}, bottomrule=.15mm, rightrule=.15mm, arc=.35mm, toprule=.15mm, colback=white, opacityback=0, leftrule=.75mm, coltitle=black, opacitybacktitle=0.6]

The U.S. government conducts a Census of Agriculture every five years,
collecting data on all farms (defined as any place from which \$1000 or
more of agricultural products were produced and sold). The file
\texttt{agpop.csv} (textbook data) contains historical information from
1982, 1987, and 1992 on the number of farms, total acreage devoted to
farms, number of farms with fewer than 9 acres, and number of farms with
more than 1000 acres for the population consisting of the \(N=3078\)
counties and county-equivalents in the United States.

Draw a sample of 300 farms without replacement using both
\texttt{sample} and \texttt{srswor}. Save one of these data frames with
the name \texttt{ag.srs} for later use.

\end{tcolorbox}

\hypertarget{formulas-for-estimation}{%
\section{Formulas for Estimation}\label{formulas-for-estimation}}

Below is a table of common statistics and how they are estimated under
the SRS framework. This table can also be found on the
\href{https://sampling-458.netlify.app/notes/formulas.html}{formulas}
page.

\begin{longtable}[]{@{}
  >{\raggedright\arraybackslash}p{(\columnwidth - 4\tabcolsep) * \real{0.2500}}
  >{\raggedright\arraybackslash}p{(\columnwidth - 4\tabcolsep) * \real{0.3472}}
  >{\raggedright\arraybackslash}p{(\columnwidth - 4\tabcolsep) * \real{0.4028}}@{}}
\toprule()
\begin{minipage}[b]{\linewidth}\raggedright
Measure
\end{minipage} & \begin{minipage}[b]{\linewidth}\raggedright
Unbiased Estimate \((\hat{\theta})\)
\end{minipage} & \begin{minipage}[b]{\linewidth}\raggedright
Estimated variance of \((\hat{\theta})\)
\end{minipage} \\
\midrule()
\endhead
Mean & \(\bar{y} = \frac{1}{n}\sum_{i\in S} y_{i}\) &
\(\hat{V}(\bar{y}) = (1-\frac{n}{N})\frac{s^{2}}{n}\) \\
Total & \(\hat{\tau} = N\bar{y}\) &
\(\hat{V}(\hat{\tau}) = N^{2}\hat{V}(\bar{y})\) \\
Proportion & \(\hat{p} = \bar{y}\) &
\(\hat{V}(\hat{p}) = (1-\frac{n}{N})\frac{\hat{p}(1-\hat{p})}{n-1}\) \\
\bottomrule()
\end{longtable}

\begin{itemize}
\tightlist
\item
  \(i \in S\) : Unit \(i\) is an element in the sample \(S\)
\end{itemize}

Did you notice something different about the formula for the variances?

\begin{tcolorbox}[enhanced jigsaw, toptitle=1mm, breakable, colframe=quarto-callout-important-color-frame, colbacktitle=quarto-callout-important-color!10!white, left=2mm, titlerule=0mm, bottomtitle=1mm, title=\textcolor{quarto-callout-important-color}{\faExclamation}\hspace{0.5em}{Finite Population Correction}, bottomrule=.15mm, rightrule=.15mm, arc=.35mm, toprule=.15mm, colback=white, opacityback=0, leftrule=.75mm, coltitle=black, opacitybacktitle=0.6]

\[\Big(.-\frac{.}{.}\Big)\]

The larger \% of the population that you include in your sample
(sampling fraction = \(\frac{n}{N}\)), the closer you are to a census,
the smaller the variability your estimate will have.

\end{tcolorbox}

\begin{itemize}
\tightlist
\item
  Most samples that are taken from a very large population, the fpc is
  close to \_\_\_\_.
\item
  So the variance is more determined by \_\_\_\_\_\_\_\_\_\_\_, not the
  \% of the \_\_\_\_\_\_\_\_\_\_\_.
\end{itemize}

\begin{tcolorbox}[enhanced jigsaw, toptitle=1mm, breakable, colframe=quarto-callout-tip-color-frame, colbacktitle=quarto-callout-tip-color!10!white, left=2mm, titlerule=0mm, bottomtitle=1mm, title=\textcolor{quarto-callout-tip-color}{\faLightbulb}\hspace{0.5em}{Example}, bottomrule=.15mm, rightrule=.15mm, arc=.35mm, toprule=.15mm, colback=white, opacityback=0, leftrule=.75mm, coltitle=black, opacitybacktitle=0.6]

Calculate the estimated standard deviation of the sample mean
\(\sqrt{\hat{V}(\bar{y})}\) of the the number of acres devoted to farms
in 1992 (variable \texttt{acres92}). Interpret this number.

\end{tcolorbox}

\begin{Shaded}
\begin{Highlighting}[]
\NormalTok{y.bar }\OtherTok{\textless{}{-}} \FunctionTok{mean}\NormalTok{()}
\end{Highlighting}
\end{Shaded}

\begin{verbatim}
Error in mean.default(): argument "x" is missing, with no default
\end{verbatim}

\begin{Shaded}
\begin{Highlighting}[]
\NormalTok{s2 }\OtherTok{\textless{}{-}} \FunctionTok{sum}\NormalTok{()}\SpecialCharTok{/}\NormalTok{(n}\DecValTok{{-}1}\NormalTok{)}
\end{Highlighting}
\end{Shaded}

\begin{verbatim}
Error in eval(expr, envir, enclos): object 'n' not found
\end{verbatim}

\begin{Shaded}
\begin{Highlighting}[]
\NormalTok{(sd.ybar }\OtherTok{\textless{}{-}} \FunctionTok{sqrt}\NormalTok{())}
\end{Highlighting}
\end{Shaded}

\begin{verbatim}
Error in sqrt(): 0 arguments passed to 'sqrt' which requires 1
\end{verbatim}

Sample means generated from samples of size \_\_\_ will vary from sample
to sample by \_\_\_\_\_\_\_\_\_\_\_\_\_\_\_ acres.

\begin{tcolorbox}[enhanced jigsaw, toptitle=1mm, breakable, colframe=quarto-callout-warning-color-frame, colbacktitle=quarto-callout-warning-color!10!white, left=2mm, titlerule=0mm, bottomtitle=1mm, title={You try it}, bottomrule=.15mm, rightrule=.15mm, arc=.35mm, toprule=.15mm, colback=white, opacityback=0, leftrule=.75mm, coltitle=black, opacitybacktitle=0.6]

Draw a sample of size 500, and estimate the standard deviation of the
sample proportion of the number of farms with less than 200,000 acres.

\end{tcolorbox}

\begin{tcolorbox}[enhanced jigsaw, breakable, colframe=quarto-callout-important-color-frame, rightrule=.15mm, arc=.35mm, bottomrule=.15mm, colback=white, opacityback=0, leftrule=.75mm, toprule=.15mm, left=2mm]

\textbf{Clean up}\vspace{2mm}

As we go through these notes, we create a lot of the same named objects,
like \texttt{n} and \texttt{N}, or \texttt{idx} and \texttt{idx2}. Every
once in a while its a good idea to clean out your global environment to
not get confused. Use the \texttt{rm()} function to remove everything
except the population \texttt{ag}, our sample of 300 \texttt{ag.srs},
and their respective sample sizes: \texttt{n} and \texttt{N}.

\begin{Shaded}
\begin{Highlighting}[]
\FunctionTok{rm}\NormalTok{()}
\end{Highlighting}
\end{Shaded}

\end{tcolorbox}

\hypertarget{sampling-weights-lohr-ch-2.4}{%
\section{Sampling Weights (Lohr Ch
2.4)}\label{sampling-weights-lohr-ch-2.4}}

Recall that a goal of sampling is to obtain a representative sample, one
that is similar to the true unknown population. Thus, conceptually if we
duplicate certain units from our sample a certain amount of times, we
could ``reconstruct'' what the population looks like. That is, we could
create \(w_{i}\) copies of unit \(i\) for each unit in the sample.

\begin{tcolorbox}[enhanced jigsaw, toptitle=1mm, breakable, colframe=quarto-callout-important-color-frame, colbacktitle=quarto-callout-important-color!10!white, left=2mm, titlerule=0mm, bottomtitle=1mm, title=\textcolor{quarto-callout-important-color}{\faExclamation}\hspace{0.5em}{Definition: Sampling Weight (Design weight)}, bottomrule=.15mm, rightrule=.15mm, arc=.35mm, toprule=.15mm, colback=white, opacityback=0, leftrule=.75mm, coltitle=black, opacitybacktitle=0.6]

Inverse of the inclusion/selection probability for unit \(i\).

\[w_{i} = \frac{1}{\delta_i}\]

Also interpreted as the number of population units represented by unit
\(i\).

\end{tcolorbox}

In an SRS, each unit has an inclusion probability of \$\delta\_\{i\} =
\$, so the sampling weights are all \(w_{i} =\).

We don't \emph{actually} make \(w_{i}\) copies of record \(i\), but use
these weights as a multiplier in our estimation calculations.

\begin{longtable}[]{@{}
  >{\centering\arraybackslash}p{(\columnwidth - 4\tabcolsep) * \real{0.3433}}
  >{\centering\arraybackslash}p{(\columnwidth - 4\tabcolsep) * \real{0.2836}}
  >{\centering\arraybackslash}p{(\columnwidth - 4\tabcolsep) * \real{0.3731}}@{}}
\toprule()
\begin{minipage}[b]{\linewidth}\centering
Population size
\end{minipage} & \begin{minipage}[b]{\linewidth}\centering
Total
\end{minipage} & \begin{minipage}[b]{\linewidth}\centering
Mean
\end{minipage} \\
\midrule()
\endhead
\(\hat{N} = \sum_{i \in S}w_{i}\) &
\(\hat{\tau} = \sum_{i \in S}w_{i}y_{i}\) &
\(\bar{y} = \frac{\hat{\tau}}{\hat{N}}\) \\
\bottomrule()
\end{longtable}

These weighted estimators are used in all probability sampling designs.

\begin{tcolorbox}[enhanced jigsaw, toptitle=1mm, breakable, colframe=quarto-callout-tip-color-frame, colbacktitle=quarto-callout-tip-color!10!white, left=2mm, titlerule=0mm, bottomtitle=1mm, title=\textcolor{quarto-callout-tip-color}{\faLightbulb}\hspace{0.5em}{Example: Calculating weighted estimates}, bottomrule=.15mm, rightrule=.15mm, arc=.35mm, toprule=.15mm, colback=white, opacityback=0, leftrule=.75mm, coltitle=black, opacitybacktitle=0.6]

Estimate the total and average number of acres devoted to farms in 1992
using both weighted and unweighted estimates. Then compare these values
to the parameter.

\end{tcolorbox}

The sampling weights are \(w_{i} = \frac{.}{.}\) for each unit \(i\) in
the sample, so we'll add that on as a new column before calculating the
weighted estimates.

\begin{Shaded}
\begin{Highlighting}[]
\NormalTok{ag.srs}\SpecialCharTok{$}\NormalTok{wt }\OtherTok{\textless{}{-}} 
\NormalTok{(N.hat }\OtherTok{\textless{}{-}} \FunctionTok{sum}\NormalTok{()) }
\end{Highlighting}
\end{Shaded}

\begin{verbatim}
Error in ag.srs$wt <- (N.hat <- sum()): object 'ag.srs' not found
\end{verbatim}

Calculate weighted and unweighted estimates, then the pop parameters.

\begin{Shaded}
\begin{Highlighting}[]
\NormalTok{tau.hat.wt }\OtherTok{\textless{}{-}} 
\NormalTok{y.bar.wt   }\OtherTok{\textless{}{-}} 
\NormalTok{tau.hat.nowt }\OtherTok{\textless{}{-}} 
\NormalTok{y.bar.nowt   }\OtherTok{\textless{}{-}} 
\NormalTok{mu  }\OtherTok{\textless{}{-}} 
\NormalTok{tau }\OtherTok{\textless{}{-}} 
\end{Highlighting}
\end{Shaded}

\begin{verbatim}
Error: <text>:7:0: unexpected end of input
5: mu  <- 
6: tau <- 
  ^
\end{verbatim}

Package it in a data frame for easier viewing.

\begin{Shaded}
\begin{Highlighting}[]
\FunctionTok{data.frame}\NormalTok{(}
  \AttributeTok{Measure =} \FunctionTok{c}\NormalTok{(}\StringTok{"Total"}\NormalTok{, }\StringTok{"Mean"}\NormalTok{), }
  \AttributeTok{Parameter =} \FunctionTok{c}\NormalTok{(tau,mu),}
  \AttributeTok{Unweighted =} \FunctionTok{c}\NormalTok{(tau.hat.nowt, y.bar.nowt), }
  \AttributeTok{Weighted =} \FunctionTok{c}\NormalTok{(tau.hat.wt, y.bar.wt)}
\NormalTok{  ) }\SpecialCharTok{|\textgreater{}}\NormalTok{ knitr}\SpecialCharTok{::}\FunctionTok{kable}\NormalTok{(}\AttributeTok{align =} \StringTok{\textquotesingle{}lccc\textquotesingle{}}\NormalTok{, }
             \AttributeTok{caption =} \StringTok{"Comparing weighted and unweighted estimates"}\NormalTok{)}
\end{Highlighting}
\end{Shaded}

\begin{verbatim}
Error in data.frame(Measure = c("Total", "Mean"), Parameter = c(tau, mu), : object 'tau' not found
\end{verbatim}

\begin{tcolorbox}[enhanced jigsaw, toptitle=1mm, breakable, colframe=quarto-callout-warning-color-frame, colbacktitle=quarto-callout-warning-color!10!white, left=2mm, titlerule=0mm, bottomtitle=1mm, title={You try it}, bottomrule=.15mm, rightrule=.15mm, arc=.35mm, toprule=.15mm, colback=white, opacityback=0, leftrule=.75mm, coltitle=black, opacitybacktitle=0.6]

Calculate the proportion of farms with less than 200k acres, with, and
without weights. Compare to the population proportion.

\end{tcolorbox}

\hypertarget{analysis-using-the-survey-package-lohr-ch-2.6}{%
\section{\texorpdfstring{Analysis using the \texttt{survey} package
(Lohr Ch
2.6)}{Analysis using the survey package (Lohr Ch 2.6)}}\label{analysis-using-the-survey-package-lohr-ch-2.6}}

Survey designs are specified using the \texttt{svydesign} function. The
main arguments to the the function are id to specify sampling units,
weights to specify sampling weights, and fpc to specify finite
population size corrections. These arguments should be given as
formulas, referring to columns in a data frame given as the data
argument.

\begin{Shaded}
\begin{Highlighting}[]
\NormalTok{ag.srs.dsgn }\OtherTok{\textless{}{-}} \FunctionTok{svydesign}\NormalTok{(}\AttributeTok{id =} 
                         \AttributeTok{weights =} 
                         \AttributeTok{fpc =} 
                         \AttributeTok{data =} 
\NormalTok{                           )}
\end{Highlighting}
\end{Shaded}

\begin{verbatim}
Error: <text>:2:34: unexpected '='
1: ag.srs.dsgn <- svydesign(id = 
2:                          weights =
                                    ^
\end{verbatim}

The \texttt{id} argument specifies the clusters. We are not using any
clustering, so each unit has it's own id.

\hypertarget{point-and-interval-estimates}{%
\subsection{Point and interval
estimates}\label{point-and-interval-estimates}}

Survey design adjusted estimates can be obtained using \texttt{svymean}
and \texttt{svytotal}.

\begin{Shaded}
\begin{Highlighting}[]
\FunctionTok{svymean}\NormalTok{()}
\end{Highlighting}
\end{Shaded}

\begin{verbatim}
Error in svymean(): could not find function "svymean"
\end{verbatim}

\begin{Shaded}
\begin{Highlighting}[]
\FunctionTok{svytotal}\NormalTok{()}
\end{Highlighting}
\end{Shaded}

\begin{verbatim}
Error in svytotal(): could not find function "svytotal"
\end{verbatim}

Estimates for multiple variables can be obtained at the same time.

\begin{Shaded}
\begin{Highlighting}[]
\FunctionTok{svymean}\NormalTok{(}\SpecialCharTok{\textasciitilde{}}\NormalTok{\_\_\_\_\_\_\_}\SpecialCharTok{+}\NormalTok{\_\_\_\_\_\_\_, ag.srs.dsgn)}
\end{Highlighting}
\end{Shaded}

\begin{verbatim}
Error: <text>:1:11: unexpected input
1: svymean(~__
              ^
\end{verbatim}

You can pass the results to construct confidence intervals.

\begin{Shaded}
\begin{Highlighting}[]
\FunctionTok{\_\_\_\_\_\_\_}\NormalTok{(}\SpecialCharTok{\textasciitilde{}}\NormalTok{acres92}\SpecialCharTok{+}\NormalTok{farms92, ag.srs.dsgn) }\SpecialCharTok{|\textgreater{}} \FunctionTok{\_\_\_\_\_\_\_}\NormalTok{()}
\end{Highlighting}
\end{Shaded}

\begin{verbatim}
Error: <text>:1:2: unexpected input
1: __
     ^
\end{verbatim}

\begin{quote}
interpretation here
\end{quote}

\begin{tcolorbox}[enhanced jigsaw, toptitle=1mm, breakable, colframe=quarto-callout-warning-color-frame, colbacktitle=quarto-callout-warning-color!10!white, left=2mm, titlerule=0mm, bottomtitle=1mm, title={You try it}, bottomrule=.15mm, rightrule=.15mm, arc=.35mm, toprule=.15mm, colback=white, opacityback=0, leftrule=.75mm, coltitle=black, opacitybacktitle=0.6]

Using your sample of 300 farms, estimate the total number of farms in
the United States in 1987 and the average farm size in acres for the
same year. Report both point and interval estimates.

\end{tcolorbox}



\end{document}
